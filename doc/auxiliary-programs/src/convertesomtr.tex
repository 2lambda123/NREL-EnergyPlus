\chapter{convertESOMTR}\label{convertesomtr}

This simple post processing utility will convert the raw data ``ESO'' and ``MTR'' files to IP (Inch-Pound) units before later processing into CSV files. EP-Launch has an option to automatically convert to IP units that invokes convertESOMTR, see VIEW - Options - Miscellaneous dialog box. The ReadVarsESO program will take these converted files and make them into normal CSV files but will have IP units. The RunEPlus batch file does not include this option but could be edited to perform the same functions if desired. If OutputControl:Files is used to write CSV output (variables and meters) directly from EnergyPlus, convertESOMTR and ReadVarsESO will be skipped when running with EP-Launch and RunEplus.bat.

Technically speaking, the convertESOMTR program uses the ``convert.txt'' file which contains the conversion factors. It creates files ``ip.eso'' and ``ip.mtr'' as appropriate. The batch examples then renames the old eplusout.eso to eplusout.esoold, old eplusout.mtr to eplusout.mtrold and the ip files to the default eplusout.eso, eplusout.mtr.

The convert.txt file contains the conversion factors using three different commands.

conv,\textless{}si-unit\textgreater{},\textless{}ip-unit\textgreater{},\textless{}multiplier\textgreater{},\textless{}offset\textgreater{}

wild,\textless{}match-string\textgreater{},\textless{}si-unit\textgreater{},\textless{}ip-unit\textgreater{}

vari,\textless{}variable-name-no-units\textgreater{},\textless{}si-unit\textgreater{},\textless{}ip-unit\textgreater{}

If a specific variable needs to be converted, the `vari' line may be used to convert the units on that specific variable only. To convert a class of variables that contains a specific string of characters in the names of the variables, the `wild' line may be used. The `conv' lines are the lines that actually create the conversion factors. If no `vari' or `wild' match a variable, then it is converted used the first `conv' line that matches. The default convert.txt file contains some conversions for Inch-Pound units but any set of units may be used by editing the convert.txt file. Note that the convert.txt file uses the standard EnergyPlus comment character (!).

A snippet of the convert.txt file:

\begin{lstlisting}
! Power
!------------------------------
!    (1 kW / 1000 W)
conv,W,kW,0.001,0
!    (1 Btuh/ 0.2928751 W) * (1 kBtuh/1000 Btuh)
conv,W,kBtuh,3.41442E-03,0
\end{lstlisting}
