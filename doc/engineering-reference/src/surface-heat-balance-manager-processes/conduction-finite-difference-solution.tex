\section{Conduction Finite Difference Solution Algorithm}\label{conduction-finite-difference-solution-algorithm}

\subsection{Basic Finite Difference Solution Approach}\label{basic-finite-difference-solution-approach}

EnergyPlus models follow fundamental heat balance principles very closely in
almost all aspects of the program. However, the simulation of building surface
constructions has relied on a conduction transfer function (CTF) transformation
carried over from BLAST. This has all the usual restrictions of a
transformation-based solution: constant properties, fixed values of some
parameters, and do not produce results for the interior of the surface. As the
energy analysis field moves toward simulating more advanced constructions, such
as phase change materials (PCM), it becomes necessary to step back from
transformations to more fundamental forms. Accordingly, a conduction finite
difference (CondFD) solution algorithm has been incorporated into EnergyPlus.
This does not replace the CTF solution algorithm, but complements it for cases
where the user needs to simulate phase change materials or variable thermal
conductivity. It is also possible to use the finite difference algorithm for
zone time steps as short as one minute.

EnergyPlus includes two different options for the specific scheme or formulation
used for the finite difference model. The first scheme is referred to as Crank-
Nicholson and was the formulation used in EnergyPlus prior to version 7. As of
version 7 a second scheme was added and is referred to as fully implicit. The
selection between the two can be made by the user with the
HeatBalanceSettings:ConductionFiniteDifference input object. Once selected, the
same scheme is used throughout the simulation. Although the two different schemes
differ in their fundamental heat transfer equations, they share nearly all the
same supporting models for material properties, data storage, solution schemes,
and spatial discretization algorithms.

The Crank-Nicholson scheme is semi-implicit and based on an Adams-Moulton
solution approach. It is considered second-order in time. The algorithm uses an
implicit finite difference scheme coupled with an enthalpy-temperature function
to account for phase change energy accurately. The implicit formulation for an
internal node is shown in the equation below.

\begin{equation}
C_p \rho \Delta x \frac{T_i^{j+1}-T_i^j}{\Delta t} = 
     \frac{1}{2}\left(k_W\frac{T_{i+1}^{j+1}-T_{i}^{j+1}}{\Delta x} +
                      k_E\frac{T_{i-1}^{j+1}-T_{i}^{j+1}}{\Delta x} + 
                      k_W\frac{T_{i+1}^{j}-T_{i}^{j}}{\Delta x} +
                      k_E\frac{T_{i-1}^{j}-T_{i}^{j}}{\Delta x}\right)
\label{eq:InternalNodeImplicitEquation}
\end{equation}
%
where:
\begin{wherelist}
\item[T] node temperature
\item[\Delta t] calculation time step
\item[\Delta x] finite difference layer thickness (always less 
than construction layer thickness)
\item[C_p] specific heat of material
\item[k_W] thermal conductivity for interface between node $i$ and node $i+1$
\item[k_E] thermal conductivity for interface between node $i$ and node $i-1$
\item[\rho] density of material
\end{wherelist}
%
with superscripts and subscripts:
%
\begin{wherelist}
\item[i] node being modeled
\item[i+1] adjacent node to interior of construction
\item[i-1] adjacent node to exterior of construction
\item[j+1] new time step
\item[j] previous time step
\end{wherelist}
%
Then, this equation is accompanied by a second equation that relates enthalpy
and temperature.

\begin{equation}
h_i = \text{HTF}\left(T_i\right)
\end{equation}
%
where HTF is an enthalpy-temperature function that uses user input data.

The fully implicit scheme is also based on an Adams-Moulton solution approach.
It is considered first order in time. The model equation for this scheme is
shown in the following equation.

\begin{equation}
C_p\rho \Delta x \frac{T_i^{j + 1} - T_i^j}{\Delta t}
 = \left( k_W\frac{T_{i + 1}^{j + 1} - T_i^{j + 1}}{\Delta x}
 + k_E\frac{T_{i - 1}^{j + 1} - T_i^{j + 1}}{\Delta x} \right)
\end{equation}

For both schemes, EnergyPlus uses the following four types of nodes, as shown in
the figure below (1) interior surface nodes, (2) interior nodes, (3) material
interface nodes and (4) external surface nodes. The grid for each material is
established by specifying a half node for each edge of a material and equal size
nodes for the rest of the material. Equations such as
\ref{eq:InternalNodeImplicitEquation} are formed for all nodes in a
construction. The formulation of all node types is basically the same.

\begin{figure}[hbtp] % fig 15
\centering
\includegraphics[width=0.9\textwidth, height=0.9\textheight, keepaspectratio=true]{media/image176.png}
\caption{Node depiction for Conduction Finite Difference Model \protect \label{fig:node-depiction-for-conduction-finite}}
\end{figure}

In the CondFD model, surface discretization depends on the thermal diffusivity
of the material ($\alpha$) and time step ($\Delta t$) selected, as shown in the
equation below. The default value of 3 for the space discretization constant,
\emph{C}, is the inverse of a grid Fourier Number:

\begin{equation}
Fo = \frac{\alpha \Delta t}{\Delta x^2}
\end{equation}
%
and is based on the stability requirement for the explicit mode that requires
values higher than 2, or a Fourier number lower than 0.5. However, CondFD uses
implicit schemes that do not have the same stability requirements as the
explicit mode. Thus, the default 3 was originally set rather arbitrary. As of
version 7, the value of this constant can be controlled by the user with the
input field called Space Discretization Constant in the
HeatBalanceSettings:ConductionFiniteDifference input object. The discretization
method allows CondFD to assign different node spacing or grid size to different
material layers in a wall or roof, as building walls and roofs typically consist
of several layers of different materials having different thermal properties.

\begin{equation}
\Delta x = \sqrt {\text{C}\alpha \Delta t}
\end{equation}

The actual integer number of nodes for each layer is then calculated by rounding
off the result from dividing the length of the material layer by the result of
the equation above. After this, $\Delta x$ is recalculated by dividing the
length of the material by the number of nodes. A full node is equal to two half
nodes. Lower values for the Space Discretization Constant yield more nodes, with
higher values yield fewer nodes.

Because the solution is implicit, a Gauss-Seidel iteration scheme is used to
update to the new node temperatures in the construction and under-relaxation is
used for increased stability.~ The Gauss-Seidel iteration loop is the inner-most
solver and is called for each surface.~ It is limited to 30 iterations but will
exit early when the sum of all the node temperatures changes between the last
call and the current call, normalized by the sum of the temperature values, is
below $\sim$0.000001C. This convergence criteria is typically met after 3
iterations, except when PCMs are simulated as it takes an average of 2-3 more
iterations when PCM are changing phase. If the number if iterations needed to
met convergence criteria start to increase, an automatic internal relaxation
factor stabilities the solution and in most cases keep the number of iterations
less than 10.

EnergyPlus also uses a separate, outer iteration loop across all the different
inside surface heat balances so that internal long-wave radiation exchange can
be properly solved. For CTF formulations, this iteration is controlled by a
maximum allowable temperature difference of \SI{0.002}\celsius\ for inside face
surface temperatures from one iteration to the next (or a limit of 100 iterations).
CondFD uses the same default value for allowable temperature difference as CTF.
However, this parameter was found to often need to be smaller for stability and
so the inside surface heat balance manager uses a separate allowable maximum
temperature difference when modeling CondFD. The user can control the value of
the relaxation factor by using the input field called Inside Face Surface
Temperature Convergence Criteria in the
HeatBalanceSettings:ConductionFiniteDifference input object. In addition, if
the program detects that there is instability by watching for excessive numbers
of iterations in this outer loop and may decrease the relaxation factor. Users
can also output the number of iterations inside of CondFD loop for each surface
and the outer internal heat balance loop for each zone with ``CondFD Inner
Solver Loop Iterations'' and ``Heat Balance Inside Surfaces Calculation
Iterations'' respectively.

\begin{equation}
T\textsubscript{$i$,new} = T\textsubscript{$i$,old}
                         + \text{Relax}\left(T\textsubscript{$i$,new}
                            - T\textsubscript{$i$,old} \right)
\end{equation}

Because of the iteration scheme used for CondFD, the node enthalpies get
updated each iteration, and then they are used to develop a variable Cp if a
phase change material is being simulated. This is done by including a third
equation for Cp.

For inputs with MaterialProperty:PhaseChange, the specific heat is calculated
from the tabulated input data of temperature/enthalpy pairs:

\begin{equation}
Cp = \frac{h\textsubscript{$i$,new} - h\textsubscript{$i$,old}}
          {T\textsubscript{$i$,new} - T\textsubscript{$i$,old}}
\end{equation}

For inputs with MaterialProperty:PhaseChangeHysteresis, the specific heat is not
only dependent on the current state, but also the previous state, as it captures
the hysteresis physics present between the melting and freezing processes.

\begin{equation}
Cp = f\left(T\textsubscript{$i$,new}, T\textsubscript{$i$, prev},
            \text{PhaseState}\textsubscript{new},
            \text{PhaseState}\textsubscript{prev}\right)
\end{equation}

The hysteresis model also has inputs for solid and liquid state values for
thermal conductivity and density.  Within the transition region of the
hysteresis model the average of the two is used.  If the dynamic nature of these
parameters is not known, the user may enter the same value for both states and
that static value will be used throughout.

The actual formulation of the hysteresis model is captured in
Figure~\ref{fig:PCMHysteresis}.  The inputs that led to those curves were as
follows:

\begin{itemize}
\item All temperature ``difference'' inputs were set to 1.0 as this is an
example only.  Using smaller or larger temperature differences would shrink or
expand the freezing or melting regions accordingly.
\item The total latent heat transferred during the phase transition process is
set to 25000 J/kg
\item The peak freezing temperature is set to \SI{23}\celsius, where the
freezing curve transition is centered.
\item The peak melting temperature is set to \SI{27}\celsius, where the
melting curve transition is centered.
\end{itemize}

\begin{figure}
  \centering
  \includegraphics[width=0.8\textwidth]{media/PCMModel.png}
  \caption{Hysteresis PCM Model Curves}
  \label{fig:PCMHysteresis}
\end{figure}

The iteration scheme assures that the correct enthalpy, and therefore the
correct $C_p$ is used in each time step, and the enthalpy of the material is
accounted for accurately. Of course, if the material is regular, the user input
constant $C_p$ is used.

For non-hysteresis finite difference calculations, the algorithm also has a
provision for including a temperature coefficient to modify the thermal
conductivity. The thermal conductivity is obtained from:

\begin{equation}
k = k_o + k_1\left( T_i - 20 \right)
\end{equation}
%
where:
\begin{wherelist}
\item[k_o] is the \SI{20}\celsius\ value of thermal conductivity (normal IDF
input)
\item[k_1] is the change in conductivity per degree temperature difference from
\SI{20}\celsius
\end{wherelist}

As of Version 7, the CondFD implementation was changed to evaluate the thermal
conductivity at the interface between nodes, as shown below. In this case,
EnergyPlus uses a linear interpolation between nodal points.

\begin{equation}
%\medmuskip=0mu
%\thinmuskip=0mu
%\thickmuskip=0mu
%\nulldelimiterspace=0pt
%\scriptspace=0pt
C_p\rho\Delta x\frac{T_i^{j + 1} - T_i^j}{\Delta t} = 
\frac{1}{2}\left[ \left( k_W\frac{T_{i + 1}^{j + 1} - T_i^{j + 1}}{\Delta x}
                       + k_E\frac{T_{i - 1}^{j + 1} - T_i^{j + 1}}{\Delta x}
                 \right)
                + \left( k_W\frac{T_{i + 1}^j - T_i^j}{\Delta x}
                       + k_E\frac{T_{i - 1}^j - T_i^j}{\Delta x}\right)\right]
\end{equation}
%
where,
%
\begin{equation}
k_W = \frac{k_{i + 1}^{j + 1} + k_i^{j + 1}}{2}
\end{equation}
%
and
%
\begin{equation}
k_E = \frac{k_{i - 1}^{j + 1} + k_i^{j + 1}}{2}
\end{equation}
%
These additional property information values are put into the input file as
explained in the Input/Output Reference Document, but it consists simply of a
value for k1 and set of enthalpy temperature pairs that describe the enthalpy of
the phase change material in straight line segments with respect to temperature.

A graph showing the effect of a large PCM on the outside surface of a zone is
shown below. The phase change temperature was \SI{30}\celsius, and the flat
temperature response during the phase change is obvious. This example was run
with a zone time step of one minute to show that such small time steps can be
done with the finite difference solution technique. It is more efficient to set
the zone time step shorter than those used for the CTF solution algorithm. It
should be set to 20 time steps per hour or greater, and can range up to 60. The
finite difference algorithm actually works better with shorter zone time steps.
The computation time has a minimum at a zone time step around two minutes (30
time steps/hr), and increases for shorter or longer zone time steps.

\begin{figure}[hbtp] % fig 16
\centering
\includegraphics[width=0.9\textwidth, height=0.9\textheight,
keepaspectratio=true]{media/image185.png}
\caption{Effects of Large PCM on Outside Zone Surface \protect
\label{fig:effects-of-large-pcm-on-outside-zone-surface}}
\end{figure}

\subsection{Finite Difference Node Arrangement in Surfaces}%
\label{finite-difference-node-arrangement-in-surfaces}

The Conduction Finite Difference algorithm determines the number of nodes in
each layer of the surface based on the Fourier stability criteria. The node
thicknesses are normally selected so that the time step is near the explicit
solution limit in spite of the fact that the solution is implicit. For very
thin, high conductivity layers, a minimum of two nodes is used. This means two
half thickness nodes with the node temperatures representing the inner and outer
faces of the layer. All thicker layers also have two half thickness nodes at
their inner and outer faces. These nodes produce layer interface temperatures.

The ConductionFiniteDifferenceSimplified capability was removed as of Version
7.2.

\subsection{Conduction Finite Difference Variable Thermal Conductivity}%
\label{conduction-finite-difference-variable-thermal-conductivity}

The Conduction Finite Difference algorithm has also been given the capability to
use an expanded thermal conductivity function. This function, explained in the
input/output document, is similar to the temperature enthalpy function. It
consists of pairs of temperature and thermal conductivity values that form a
linear segmented function. It is established with the
MaterialProperty:VariableThermalConductivity object.

\subsection{Conduction Finite Difference Source Sink Layers}%
\label{conduction-finite-difference-source-sink-layers}

The Conduction Finite Difference algorithm can also invoke the source/sink layer
capability by using the \textbf{ConstructionProperty:InternalHeatSource} object.

\subsection{Conduction Finite Difference Heat Flux Outputs}%
\label{conduction-finite-difference-heat-flux-outputs}

The Conduction Finite Difference algorithm can output the heat flux at each node
and the heat capacitance of each half-node. During the CondFD solution
iterations, the heat capacitance of each half node (CondFD Surface Heat
Capacitance Node \textless{} n \textgreater{}) is stored:

\begin{equation}
\text{HeatCap}_i = \frac{1}{2}C_{p,i}\Delta x_i \rho_i
\end{equation}
%
For nodes which are at the inside or outside face of the surface, there is only
one half-node.

After the CondFD node temperatures have been solved for a given timestep, the
heat fluxes (CondFD Surface Heat Flux Node \textless{} i \textgreater{} ) are
calculated beginning with the inside face of the surface.

\begin{equation}
\text{QDreport}_N = Q\textsubscript{inside}
\end{equation}
%
for the remaining nodes

\begin{align}
\begin{split}
\text{QDreport}_{i} = \text{QDreport}_{i+1}
+&\text{HeatCap1}_{i+1}\frac{T\textsubscript{$i+1$,new} -
T\textsubscript{$i+1$,old}}{\Delta t} \\
- \text{QSource}_{i}
+&\text{HeatCap2}_{i}
\frac{T\textsubscript{$i$,new}-T\textsubscript{$i$,old}}{\Delta t}
\end{split}
\end{align}
%
where:
\begin{wherelist}
\item[\text{HeatCap1}_i] heat capacitance associated with a given outer half-node
\item[\text{HeatCap2}_i] heat capacitance associated with a given inner half-node 
\item[\text{QDreport}_i] CondFD Surface Heat Flux Node \textless{} i
\textgreater{}
\item[\text{QSource}_i] internal source heat flux at node $i$
\item[Q\textsubscript{inside}] Surface Inside Face Conduction Heat Transfer Rate
per Area {[}W/m\(^{2}\){]}
\item[N] total number of nodes in a surface including the surface inside face
node.
\end{wherelist}
%
Note that the variable \emph{TotNodes} used in the source code is actually N-1.
The surface inside face node is referenced as \emph{TotNodes+1}.


\subsection{References}\label{references-013}

Pedersen C.O., Enthalpy Formulation of conduction heat transfer problems involving latent heat, Simulation, Vol 18, No. 2, February 1972

Versteeg, H. and Malalasekra, W. 1996. An introduction to computational fluid dynamic: the finite volume method approach. Prentice Hall.

Tabares-Velasco, P.C. and Griffith, B. 2012. Diagnostic Test Cases for Verifying Surface Heat Transfer Algorithms and Boundary Conditions in Building Energy Simulation Programs, Journal of Building Performance Simulation, doi:10.1080/19401493.2011.595501

Tabares-Velasco, P.C., Christensen, C. and Bianchi, M. 2012. Verification and Validation of EnergyPlus Phase Change Material Model for Opaque Wall Assemblies, Building and Environment 54: 186-196.
